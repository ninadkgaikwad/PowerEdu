\documentclass[varwidth]{standalone}


\providecommand{\packageName}[1]{PowerEdu.jl}


\usepackage{lipsum}

\begin{document}
% \begin{minipage}{0.8\linewidth}
% minipage is good for using the align environment, which otherwise does not 
% like the standalone environment. But it seems to introduce unnecessary 
% space in the main.tex file, in this case, after the abstract. 
% Ideally should test/check why that happens.
% For now, disabling minipage and hoping that nothing bad happens.
    \section{Introduction}
    Electrical power system operation requires planning and analysis in a variety
    of aspects. Traditionally, these aspects have been grouped under two realms,
    namely Power System Analysis, for quasi-steady-state studies
    such as Power Flow, Economic Dispatch, Optimal Power Flow, State Estimation, etc.
    and Power System Dynamics, which delves into time-domain, dynamic behaviours
    like transient and small signal stability studies. Aspects like Sparse Power
    Flow which requires usage of sparse data
    structures especially highlight how actual implementation can vary from
    textbook algorithms, which are often written in pseudo code.
    Our free and open-source package, \packageName{} aims to serve as a bridge for budding power system engineers who
    may find the initial stages of coding and computational analysis challenging.
    By offering an accessible, well-documented and easy to tinker platform, we
    aim to narrow the gap between newcomers to the field and seasoned experts
    who have dedicated years at renowned national laboratories or corporations,
    developing sophisticated software tools utilized
    by the industry.
% \end{minipage}
\end{document}
